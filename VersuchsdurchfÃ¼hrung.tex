\documentclass[10pt,a4paper]{article}
\usepackage[utf8]{inputenc}
\usepackage{amsmath}
\usepackage{amsfonts}
\usepackage{amssymb}
\usepackage{makeidx}
\usepackage[left=2cm,right=2cm,top=2cm,bottom=2cm]{geometry}
\author{Wolfgang Breu, Timur Numi\'{c}}
\begin{document}
\part*{Versuchsdurchführung}
\section{Untersuchung der Signale der Geräte}
Um die Unterschiedliche Arbeitsweise der beiden Szintillatortypen zu verdeutlichen, vergleichen wir beide Signale mit Hilfe eines Oszilloskopes. Dabei verwenden wir die \textit{Europium 152} Probe als Quelle.
Für die weiteren Messungen verwenden wir den NaJ-Szintillationszähler, aufgrund der besseren Eigenschaften. Zunächst untersuchen wir die Signalform nach den einzelnen NIM-Geräten, d.h. nach dem
\begin{itemize}
\item[(a)] Photomultiplier
\item[(b)] Verstärker (unipolarer und bipolarer Ausgang)
\item[(c)] Single Channel Analyzer
\item[(d)] Gate and Delay Generator
\end{itemize}
Danach bestimmen wir noch die zeitliche Verzögerung des Signals zwischen dem Eingang des Amplifiers und dem 
\begin{itemize}
\item[(a)] unipolaren Ausgang des Amplifiers
\item[(b)] bipolaren Ausgang des Amplifiers
\item[(c)] Ausgang des Single Channel Analyzers
\end{itemize}
\section{Eichung der Kanäle des Multi Channel Analyzers (MCA)}
Nun beobachten wir die Auswirkungen verschiedener Einstellungen der NIM-Geräte. Mit den gesammelten Daten können wir nun optimale Einstellungen für die Aufnahme eines Spektrums mit dem MCA vornehmen. Wir messen nun die $\gamma$-Spektren von \textit{Natrium 22, Cobalt 60, Europium 152} mit einer Messdauer von jeweils 30 min. Anhand der bekannten intensiven Linien dieser Präparate können wir nun eine vernünftige Kanaleichung des MCAs durchführen. Dazu muss jedoch zuerst eine Korrektur des Untergrundes durchgeführt werden. 
\section{Messung des Spektrums von \textit{Thorium 228}}
Für eine ausführliche Analyse nehmen wir das $\gamma$-Spektrum von \textit{Thorium 228} auf. Die Messdauer beträgt hier 180 min. Die auftretenden Peaks werden später einem Zerfallsprozess zugeordnet. Hier muss die Ausbeutekurve des Szintillators berücksichtigt werden.
\section{Untergrundsmessung}
Wir messen auch den Untergrund mit einer Messdauer von 180 min.
\section{Winkelkorrelation von "Vernichtungsphotonen"}
Mit Hilfe beider Szintillationsähler messen wir nun die Winkelkorrelation der beiden Vernichtungsphotonen. Dazu verwenden wir bei den SCAs ein Fenster, welches nur den Peak bei 511 keV registriert. Dabei wird der Winkel in Inkrementen von $5^\circ$ erhöht. Es wid ein Gesamtwinkel von $270^\circ$ abgedeckt. Anschließend wird eine Messung ohne Justage des Delays durchgeführt um die Anzahl der zufälligen Koinzidenzen zu berücksichtigen. 
\end{document}