\documentclass[11pt,a4paper]{article}
\usepackage[utf8]{inputenc}
\usepackage[english]{babel}
\usepackage{amsmath}
\usepackage{amsfonts}
\usepackage{graphicx}
\begin{document}
\part{Introduction}
In this experiment we want to detect the radioactive decay of radioactive preparations by scintillator.
\section*{Aim of the experiment}
The main goal of our measurements is, to record and interprete the $\gamma$-spectrum of Thorium 228. We have to calibrate our gear, so that we can determine specific energies. In addition to that, we determine the angle-correlation of two annihilation photons. We also gen an overview about the differences of plastic- and organic szintillators and the concept of the NIM-devices.
\section{Conduction of the experiment}
We compare the signal length of the NaJ-scintillator and plastic szntillator by using variable radioactive preparations. First we observe the signal of the NaJ-scintillator after passing the photomultiplier. Then we compare the signals after using the amplifier, at both, the unipolar and bipolar output. Also we proof the signal after passing the Single Channel Analyzer. The signals after Gate and Delay Generator will be also observed. All signals will be checked by an oscilloscope. After this, we take a qualitative observation for changing the settings of the NIM-Gadgets.
We measure the $\gamma$-spectrum of Na,Eu, and Co by measuring over a time of 30 min. The $\gamma$-spectrum measuring of the RdTH-compound will take more than 180 min. After this we measure the underground radiation with the same measuring-time of 180 min.
Via determined delay of the both scintillators we measure the angle correlation of the 511-keV annihilation photons. The number of the random coincident will be also quantified.

With the help of the recorded data of oscilloscope we dictate the timely delay of the above-named signals. The data of the $\gamma$-spectrum will be evaluated after correcting it using the underground measuring.
The calibration of the Multi Channel Analyzer will be conducted by the most intensive lines of the Natrium, Cobalt, and Europium. We also determine the most intense lines of the background radiation. Last but not least, we decide for the last measuring the optimal delay of the both scintillators.
\section{fundamental physical principles}
In the world exist unstable nucleis, which are radioactive. This mean, that they decay in other nucleis and emit by this process electrons, photons or $\alpha$-particle. This process is a statistical process, which drop it by the time. The nuclei are good saved in the atomic shell, so that outer influences like pressure, temperature doesn't matter to them. We differentiate three types of decays:

$\alpha$-decay: The mother nuclei decay a helium-atom with a charge of 2+. In classical cases the $\alpha$-decay doesn't exist. It can be only explain by quantum mechanic. 

$\beta$-decay: In this case there will be emit charges particle, like electron or positron. Two cases of this decays are known. The $\beta^{+}$-decay is not possible in the nature. The electron and the positron can only exist by linking of this two. It called positrionsiumatom. 

$\gamma$-decay: After sending-out of $\alpha$- or $\beta$-particles, there will be left a nuclei in excited state. By passing to the basic state the nuclei emits a photon.
There are two types of $\gamma$-decay. The inner conversion and the auger-effect. By the inner conversion a $\gamma$-quant will be not send. However the energy will be transport to an outer electron. This electron has an energy $E_e=E_{\gamma}-E_B$. The electron has a binding forces with the nuclei and conduct that the the electron will not have the complete energy of the $\gamma$-ray, because a certain energy will be lost for dissolving the electron in favor the binding-energy $E_B$.
When a auger-effect happens, then is called `Effect of the electron shell`. The superfluous energy will be deposit to an electron. This electron, it call the `Auger-electron`, has enough power to leave the atomic orbital. Now there are two faults.

A charged particle has a maximal energy of $E_{max,kin} = 2 m_e c^2 \beta^2 \gamma^2$. He loses the energy by excitation of other molecules in matter, when he passes the matter. A distinction is made between ionisation, bremsstrahlung and Cherenkov-ray. The energy loss by ionisation happens when the charged particle gets in interaction with the matter, which is passing. Bremsstrahlung is caused by coulomb-field of an nuclei when a charged particle is on the way to the nuclei. The Coulomb forces of the nuclei brakes the incoming charged particle. By slowing down the charged particle loss his energy in form of emitting x-rays.  
The loss energy by Cherenkov can be explained, that the velocity of charged particle can be higher than light in matter. 
The reach of a charged particle is given by the formula:
\[R = \int^{0}_{E} - \frac{dE}{\frac{dE}{dx}}\]
The intensity of a particle beam will drop in matter. The matter absorbs it. The absorption law is given by the formula:
\[I = I_0 \exp (-\mu x)\]

Photons can give their energy in three cases. First by photo effect, compton scattering and 
\end{document}